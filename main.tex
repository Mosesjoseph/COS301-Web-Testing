\documentclass[english]{article}

\usepackage{graphicx}
\usepackage{grffile}
\usepackage[T1]{fontenc}
\usepackage{babel}

\author{
	Mering, Wynand\\
	\texttt{13230795}
	\and
	Watt, Brenton\\
	\texttt{14032644}
	\and
	Loosen, Michael\\
	\texttt{14017254}
	\and
	Masibi, Mothusi\\
	\texttt{12004589}
	\and
	Tswene, Kamogelo\\
	\texttt{12163555}
	\and
	Mayimela, Moses\\
	\texttt{14019702}
	\and
	Hlatshwayo, Nontokozo\\
	\texttt{14414555}
	\and
	Masemola, Sibusiso\\
	\texttt{12270467}
	\and
	Dunkley, Nathan\\
	\texttt{14145759}
	\and
	Langa, Vukile\\
	\texttt{14035449}
}

\title{Web Testing\\
	\large Name of project\\
	\small Version: 1.0}
\date{\today}
\graphicspath{{Pictures/}}
\begin{document}
	\maketitle
	\begin{figure}[!t]
		\includegraphics[width=\linewidth]{up_logo.png}
	\end{figure}
	\pagenumbering{gobble}
	\newpage
	\tableofcontents
	\newpage
	\pagenumbering{arabic}
	
	\section{Introduction}
		\paragraph\newline
			In this document we will be providing our findings on testing the individual web interfaces of teams Alpha and Bravo in regards to the COS 301 mini-project.

	\section{The functionality that is correctly implemented - Alpha}
		\paragraph\newline			
			Due to the back-end of the project never being fully completed and integrated, it is difficult to accurately establish what was and what wasn't correctly implemented as nothing can be thoroughly tested, however, after following the readme that was placed in the repository, the website was indeed running and could be navigated.
			\newline
			\newline
			With regards to publications, there is a link to publications itself, which comes across as an add publications page. There is a sub-field in which you can change the current life-cycle state of the publication but upon testing it was established that there was only one field to choose from but the functionality was there none-the-less.The ability to change the publication title and envisaged completion date work very well and meet requirements.
			\newline
			\newline
			When navigating to the research page, the functionality to add, suspend or reactivate a research group exists and it is well set out and easy to navigate. Obviously due to the lack of a back-end there is no way to test if the object is actually being sent.
			The researcher category addition and modification pages are also implemented and are according to spec.
			\newline
			\newline
			The overall aesthetic look of the website is good and navigation is simple and easy, nothing seemingly difficult or painful to use.
			
	\section{The functionality that is correctly implemented - Bravo}
		\subsection{General Layout}
			\paragraph \newline
				The website is aesthetically pleasing and very simple to use. All available links in the navigation bar work correctly and take the user to the relevant webpages. The "Home" link takes the user to the homepage which is just a welcome page. Under the "Publications" section are two links to either add a new publication or list all current publications and both of these links work correctly. Next is a "People" section which has 3 links to either list all users, add a new user or add a new research group respectively and all these links work. The last section of the website is the "Reports" section which contains the link "Query" which generates a report for the user and this also works correctly.

		\subsection{Publications}
			\paragraph \newline
				The "List Publications" section correctly displays the publications that have been added to the system, however they appear to have been hard coded because after adding a publication the list remains the same as it was before adding it. For each publication in the list, there is a "View Publication" button and an "Edit Publication" button which both, after having been clicked, provide the user with the means to either view or edit the publication depending on which button they clicked. When viewing the publication, the user can choose to export it either as a CSV file or as a Bibtex file and both of these work correctly. When adding a new publication, the user can enter all the relevant details of that publication or they can choose to import the details from a CSV file which is working correctly.

			\subsection{People}
				\paragraph \newline
					The "List People" section displays details about the users, including which research field they are part of. For each person in the list, there is a means by which the user can view or edit their information. Both of these buttons work correctly. When editing a person, the user has the option to import the information from a CSV file which is functioning correctly. When adding a person, the user can enter all the relevant details by making use of the various fields or they can opt to once again import a CSV file. The "Adding a Research Group" section correctly provides the user to enter the details of the new research group.

			\subsection{Reports}
				\paragraph \newline
					This section allows the user to type in the details of the publication that a report must be generated for. The user must enter the publication type and the name of a person or research group that the publication is assigned to. After clicking the "Generate Report" button, the user is taken to a page which displays all of the research groups and the publications that they are part of.
					
			\section{Software Architecture Specifications Partially Complied To - Alpha}
				\paragraph \newline
				The technologies that were specified in the Architectural Requirements were as follows:
				\begin{itemize}
					\item Bower
					\item Ember CLI
					\item Node.js with npm
					\item Broccolli
					\item QUnit
					\item PhantomJS
				\end{itemize}
				All of which were involved in the implementation of the Alpha web interface. Thus, nothing can be said in regard to any of the Architecture Specifications only being partially complied to.
		
\end{document}
